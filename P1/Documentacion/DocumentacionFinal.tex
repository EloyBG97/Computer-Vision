\documentclass{article}
\title{Vision por Computador : P1}
\author{Eloy}

\usepackage{graphicx}
\usepackage{float}
\usepackage{amsmath}

\begin{document}
\maketitle
\newpage

\section*{Ejercicio 1}
\textbf{Usando las funciones de OpenCV, escriba funciones que implementen los siguientes puntos:}
\subsection*{Apartado A}
\textbf{El cálculo de la convolución de una imagen con una máscara Gaussiana 2D.
Mostrar ejemplos con distintos tamaños de máscara y valores de sigma. Valorar
resultados.}

\begin{minipage}{\linewidth}
    \centering
    \begin{minipage}{0.45\linewidth}
        \begin{figure}[H]
			\includegraphics[width=\linewidth]{Ejercicio1a/gaussiana(3,3)3.png} 
            \caption{Kernel Size: 3, $\sigma$: 3}
        \end{figure}
    \end{minipage}
    \hspace{0.05\linewidth}
    \begin{minipage}{0.45\linewidth}
        \begin{figure}[H]
            \includegraphics[width=\linewidth]{Ejercicio1a/gaussiana(3,3)4.png}
            \caption{Kernel Size: 3, $\sigma$: 4}
        \end{figure}
    \end{minipage}
    
    \begin{minipage}{0.45\linewidth}
        \begin{figure}[H]
            \includegraphics[width=\linewidth]{Ejercicio1a/gaussiana(3,3)5.png}
            \caption{Kernel Size: 3, $\sigma$: 5}
        \end{figure}
    \end{minipage}    
\end{minipage}
\linebreak
Las tres imagenes anteriores, aunque parecen ser la misma, han sido tratadas con distintos filtros gaussianos. En este caso hemos fijado el tamano del kernel a 3, y hemos variado el valor de $\sigma$. El resultado de los 3 filtros ha sido el mismo ya que el tamano del kernel no es lo suficientemente grande como para poder representar al menos el 95\% de la funcion gaussiana que define cada filtro.

\begin{minipage}{\linewidth}
    \centering
    \begin{minipage}{0.45\linewidth}
        \begin{figure}[H]
			\includegraphics[width=\linewidth]{Ejercicio1a/gaussiana(31,31)1.png} 
            \caption{Kernel Size: 31, $\sigma$: 1}
        \end{figure}
    \end{minipage}
    \hspace{0.05\linewidth}
    \begin{minipage}{0.45\linewidth}
        \begin{figure}[H]
            \includegraphics[width=\linewidth]{Ejercicio1a/gaussiana(31,31)3.png}
            \caption{Kernel Size: 31, $\sigma$: 3}
        \end{figure}
    \end{minipage}
    
    \begin{minipage}{0.45\linewidth}
        \begin{figure}[H]
            \includegraphics[width=\linewidth]{Ejercicio1a/gaussiana(31,31)5.png}
            \caption{Kernel Size: 31, $\sigma$: 5}
        \end{figure}
    \end{minipage}    
\end{minipage}
\linebreak
En el caso de estas tres imagenes, hemos repetido el experimento escogiendo un tamano de kernel mas grande. La estadistica defiende que el tamano optimo de kernel en funion de $\sigma$:

\begin{center}
$ksize = 6 \sigma + 1$
\end{center}

En este experimento hemos fijado el tamano de kernel para el maximo valor de sigma ($\sigma$ = 5). Con esto conseguimos, tal y como hemos dicho antes, representar al menos el 95\% de la funcion gaussiana.

\subsection*{Apartado B}
\textbf{Usar getDerivKernels para obtener las máscaras 1D que permiten calcular al convolución 2D con máscaras de derivadas. Representar e interpretar dichas máscaras 1D para distintos valores de $\sigma$.}

La funcion getDerivKernels genera mascaras 1D que permiten calcular la convolucion con mascaras 2D (estas mascaras 2D son separables). Tiene 3 parametros:
\begin{itemize}
\item dx $\rightarrow$ nivel de la derivada en X 
\item dy $\rightarrow$ nivel de la derivada en Y
\item ksize $\rightarrow$ el tamano del kernel
\end{itemize}

La mascara 1D para la primera derivada en X es:
\begin{center}
$$
\begin{pmatrix}
-1 & 0 & 1
\end{pmatrix}
$$
\end{center}

La mascara 1D para la segunda derivada en X es:
\begin{center}
$$
\begin{pmatrix}
1 & -2 & 1
\end{pmatrix}
$$
\end{center}

La mascara 1D para la primera derivada en Y es:
\begin{center}
$$
\begin{pmatrix}
-1 \\
0 \\
1
\end{pmatrix}
$$
\end{center}

La mascara 1D para la segunda derivada en Y es:
\begin{center}
$$
\begin{pmatrix}
1  \\
-2 \\
1
\end{pmatrix}
$$
\end{center}

Podemos observar que la diferencia entre los respectivos niveles de derivacion entre X e Y es que las mascaras de X son las transpuestas de Y.
\\

Todas las ascaras cumplen que la sumatoria de sus compoentes resulta 0.



\subsection*{Apartado C}
\textbf{Usar la funcion Laplacian para el calculo de la convolución 2D con una mascara de Laplaciana-de-Gaussiana de tamano variable. Mostrar ejemplos de funcionamiento usando dos tipos de bordes y dos valores de $\sigma$ 1 y 3.}

\begin{minipage}{\linewidth}
    \centering
    \begin{minipage}{0.45\linewidth}
        \begin{figure}[H]
			\includegraphics[width=\linewidth]{Ejercicio1c/dog(7,7)1_DEFAULT.png}             			
			\caption{Kernel Size: 7, $\sigma$: 1, Default Border}
        \end{figure}
    \end{minipage}
    \hspace{0.05\linewidth}
    \begin{minipage}{0.45\linewidth}
        \begin{figure}[H]
            \includegraphics[width=\linewidth]{Ejercicio1c/dog(19,19)3_DEFAULT.png}            
            \caption{Kernel Size: 19, $\sigma$: 3, Default Border}
        \end{figure}
    \end{minipage}
    
    \centering
    \begin{minipage}{0.45\linewidth}
        \begin{figure}[H]
            \includegraphics[width=\linewidth]{Ejercicio1c/dog(7,7)1_REFLECT.png}                        
            \caption{Kernel Size: 7, $\sigma$: 1, Reflect Border}
        \end{figure}
    \end{minipage}
    \hspace{0.05\linewidth}
    \begin{minipage}{0.45\linewidth}
        \begin{figure}[H]
            \includegraphics[width=\linewidth]{Ejercicio1c/dog(19,19)3_REFLECT.png}
            \caption{Kernel Size: 19, $\sigma$: 3, Reflect Border}
        \end{figure}
    \end{minipage}   
\end{minipage}
\linebreak

Para hacer cada una de las imagenes, hemos aplicado un filtro Laplaciano sobre un filtro Gaussiano.
\\

El filtro Laplaciano, resalta los cambios de intensidad en la imagen (aristas), mientras que el Gaussiano, suaviza la imagen para eliminar ruidos.
\\

En primer lugar, vamos a observar las diferencias entre las imagenes debido al valor de $\sigma$ escogido (notese que el el tamano del kernel es el optimo para cada $\sigma$).
\\

A mayor sigma, mas se suaviza la imagen, por tanto la diferencia de intensidades entre los pixeles van decreciendo. Es por este motivo por lo que la imagen calculada para $\sigma$ igual a 1, tiene las aristas más marcadas que la imagen calculada con $\sigma$ igual a 3.
\\

\newpage
\section*{Ejercicio 2}
\textbf{Implementar apoyandose en las funciones getDerivKernels, getGaussianKernel, pyrUp(), pyrDown(), escribir funciones los siguientes}

\subsection*{Apartado A}
\textbf{El calculo de la convolución 2D con una máscara separable de tamano variable. Usar bordes reflejados. Mostrar resultados}

\begin{minipage}{\linewidth}
    \centering
    \begin{minipage}{0.45\linewidth}
        \begin{figure}[H]
			\includegraphics[width=\linewidth]{Ejercicio2a/cat(3,3)_REFLECT.png}             			
			\caption{Kernel Size: 3,  Reflect Border}
        \end{figure}
    \end{minipage}   
\end{minipage}

\subsection*{Apartado B}
\textbf{El calculo de la convolucion 2D con una mascara 2D de 1a derivada de tamano variable. Mostrar ejemplos de funcionamiento usando bordes a cero.}

\begin{minipage}{\linewidth}
    \centering
    \begin{minipage}{0.45\linewidth}
        \begin{figure}[H]
			\includegraphics[width=\linewidth]{Ejercicio2b/cat(3,1,1)_DEFAULT.png}             			
			\caption{Kernel Size: 3, dx = 1, dx = 1,  Default Border}
        \end{figure}
    \end{minipage}   
\end{minipage}

\subsection*{Apartado C}
\textbf{El calculo de la convolucion 2D con una máscara 2D de 2ª derivada de tamano variable.}

\begin{minipage}{\linewidth}
    \centering
    \begin{minipage}{0.45\linewidth}
        \begin{figure}[H]
			\includegraphics[width=\linewidth]{Ejercicio2c/cat(3,2,2)_DEFAULT.png}             			
			\caption{Kernel Size: 3, dx = 1, dx = 1,  Default Border}
        \end{figure}
    \end{minipage}   
\end{minipage}

\subsection*{Apartado D}
\textbf{Una funcion que genere una representacion en piramide Gaussiana de 4 niveles de una imagen. Mostrar ejemplos de funcionamiento usando bordes}

\begin{minipage}{\linewidth}
    \centering
    \begin{minipage}{0.45\linewidth}
        \begin{figure}[H]
			\includegraphics[width=\linewidth]{Ejercicio2d/cat0.png}          
			\caption{Original, $\sigma$: 3}
        \end{figure}
    \end{minipage}
    \hspace{0.05\linewidth}
    \begin{minipage}{0.45\linewidth}
        \begin{figure}[H]
            \includegraphics[width=\linewidth]{Ejercicio2d/cat1.png}          
			\caption{Level 1, $\sigma$: 3}
        \end{figure}
    \end{minipage}
    
    \centering
    \begin{minipage}{0.45\linewidth}
        \begin{figure}[H]
            \includegraphics[width=\linewidth]{Ejercicio2d/cat2.png}          
			\caption{Level 2, $\sigma$: 3}
        \end{figure}
    \end{minipage}
    \hspace{0.05\linewidth}
    \begin{minipage}{0.45\linewidth}
        \begin{figure}[H]
            \includegraphics[width=\linewidth]{Ejercicio2d/cat3.png}          
			\caption{Level 3, $\sigma$: 3}
        \end{figure}
    \end{minipage}   
    
     \begin{minipage}{0.45\linewidth}
        \begin{figure}[H]
            \includegraphics[width=\linewidth]{Ejercicio2d/cat4.png}          
			\caption{Level 4, $\sigma$: 3}
        \end{figure}
    \end{minipage}
\end{minipage}


\subsection*{Apartado E}
\textbf{Una función que genere una representación en pirámide Laplaciana de 4 niveles de una imagen. Mostrar ejemplos de funcionamiento usando bordes.}


\begin{minipage}{\linewidth}
    \centering
    \begin{minipage}{0.45\linewidth}
        \begin{figure}[H]
			\includegraphics[width=\linewidth]{Ejercicio2e/cat0.png}          
			\caption{Original, $\sigma$: 3}
        \end{figure}
    \end{minipage}
    \hspace{0.05\linewidth}
    \begin{minipage}{0.45\linewidth}
        \begin{figure}[H]
            \includegraphics[width=\linewidth]{Ejercicio2e/cat1.png}          
			\caption{Level 1, $\sigma$: 3}
        \end{figure}
    \end{minipage}
    
    \centering
    \begin{minipage}{0.45\linewidth}
        \begin{figure}[H]
            \includegraphics[width=\linewidth]{Ejercicio2e/cat2.png}          
			\caption{Level 2, $\sigma$: 3}
        \end{figure}
    \end{minipage}
    \hspace{0.05\linewidth}
    \begin{minipage}{0.45\linewidth}
        \begin{figure}[H]
            \includegraphics[width=\linewidth]{Ejercicio2e/cat3.png}          
			\caption{Level 3, $\sigma$: 3}
        \end{figure}
    \end{minipage}   
    
     \begin{minipage}{0.45\linewidth}
        \begin{figure}[H]
            \includegraphics[width=\linewidth]{Ejercicio2e/cat4.png}          
			\caption{Level 4, $\sigma$: 3}
        \end{figure}
    \end{minipage}
\end{minipage}

\section*{Ejercicio 3}
Mezclando adecuadamente una parte de las frecuencias altas de una imagen con una parte de las frecuencias bajas de otra imagen, obtenemos una imagen híbrida que admite distintas interpretaciones a distintas distancias..
Para seleccionar la parte de frecuencias altas y bajas que nos quedamos de cada una de las imágenes usaremos el parámetro sigma del nucleo/máscara de alisamiento gaussiano que usaremos. A mayor valor de sigma mayor eliminación de altas frecuencias en la imagen convolucionada. Para una buena implementación elegir dicho valor de forma separada para cada una de las dos imágenes. Recordar que las máscaras 1D siempre deben tener de longitud un número impar.
Implementar una función que genere las imágenes de baja y alta frecuencia a partir de las parejas de imágenes. El valor de sigma más adecuado para cada pareja habrá que encontrarlo por experimentación
\subsubsection{Apartado A} Escribir una función que muestre las tres imágenes ( alta, baja e híbrida) en una misma ventana. (Recordar que las imágenes después de una convolución contienen número flotantes que pueden ser positivos y negativos)
\subsubsection{Apartado B} Realizar la composición con al menos 3 de las parejas de
imágenes

\begin{minipage}{\linewidth}
    \centering
    \begin{minipage}{0.45\linewidth}
        \begin{figure}[H]
			\includegraphics[width=\linewidth]{Ejercicio3/hybrid1.png}          
			\caption{Fish + Submarine}
        \end{figure}
    \end{minipage}
    \hspace{0.05\linewidth}
    \begin{minipage}{0.45\linewidth}
        \begin{figure}[H]
			\includegraphics[width=\linewidth]{Ejercicio3/hybrid2.png}          
			\caption{Einstein + Marilyn}
		\end{figure}
    \end{minipage}
    
    \centering
    \begin{minipage}{0.45\linewidth}
        \begin{figure}[H]
            \includegraphics[width=\linewidth]{Ejercicio3/hybrid3.png}          
			\caption{Plane + Bird}
        \end{figure}
    \end{minipage}
    \hspace{0.05\linewidth}
    \begin{minipage}{0.45\linewidth}
        \begin{figure}[H]
            \includegraphics[width=\linewidth]{Ejercicio3/hybrid4.png}          
			\caption{Dog + Cat}
        \end{figure}
    \end{minipage}   
    
     \begin{minipage}{0.45\linewidth}
        \begin{figure}[H]
            \includegraphics[width=\linewidth]{Ejercicio3/hybrid5.png}          
			\caption{Motorcycle + Bicycle}
        \end{figure}
    \end{minipage}
\end{minipage}


\end{document}